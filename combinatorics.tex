\documentclass{article}
\usepackage{amsmath, amssymb, amsthm}
\usepackage[shortlabels]{enumitem}
\begin{document}

\begin{center}
	\textbf{\large 100 Combinatorics Problems} \\
    \large Jake Ryder Gameroff 
\end{center}
\noindent \textbf{Q1.} Let $G$ be $k$-connected and suppose $H$ is a graph obtained from $G$ by appending a new vertex $v$ adjacent to at least $k$ vertices in $G$. Show that $H$ is $k$-connected.
\begin{proof}
Consider a subset $X \subseteq V(H)$ with $|X| < k$. If $v \notin X$, then $H \setminus X$ is connected since $G$ is $k$-connected. Otherwise, $v \in X$; by the $k$-connectivity of $G$ and since $v$ has a neighbour not in $X$ (as $\deg v \geq k$), it follows that $F = H \setminus (X \setminus v)$ is connected. Then $F \setminus v = G \setminus (X \setminus v) = H \setminus X $ is connected by the choice of $v$.
\end{proof}
\noindent \textbf{Q2.} Let \( |V(G)| \geq 3 \). Show that $G$ is 2-connected if and only if for all vertices $x,y,z$ there is a path between $x$ and $y$ containing $z$.
\begin{proof}
For ``$\Rightarrow$", suppose $G$ is 2-connected. Define \( R = \{ x,y \}  \) and \( Q = \{ z \}  \). Since \( G \) has no separation of order 2, Menger's theorem implies that there are two vertex disjoint paths (other than at \( z \)) with ends in \( R \) and \( Q \). So \( P_1 \cup P_2 \) is a path between \( x \) and \( y \) containing \( z \).

For ``$\Leftarrow$", we prove the contrapositive. If \( G \) is not 2-connected then there is a cut vertex \( \ell \). So there are vertices \( s,t \) such that there is no path between \( s \) and \( t \) in \( G \setminus \ell \). So there can be no path between \( s \) and \( \ell \) containing \( t \), else there is a path between \( s \) and \( t \) in \( G \setminus \ell \).
\end{proof}
\noindent \textbf{Q3.} Let \( k \geq 2 \). Show that if \( G \) is \( k \)-connected, then every \( k \) vertices are contained in a cycle.
\begin{proof}
	For \( k = 2 \), the result immediately follows from Menger's theorem. Now fix \( k \geq 3 \) and let \( X = \{ x_1, x_2, \hdots , x_{k}  \}  \) be a set of \( k \) vertices in \( G \). By induction, there is a cycle \( C \) containing \( \{ x_1, x_2, \hdots , x_{k-1}  \}  \). Let \( x, y \in \{ x_1, x_2, \hdots , x_{k-1}  \}  \) be such that no vertex in \( \{ x_1, x_2, \hdots , x_{k-1}  \}  \) lies between them in \( C \). Then, Menger's theorem implies that there are \( k \) internally disjoint paths with ends in \( \{ x, y \}  \) and \( \{ x_{k}  \}  \). So \( C \cup P_{x} \cup P_{y}   \) is  the desired cycle, where \( P_{x}  \) and \( P_{y}  \) are paths from \( x \) to \( x_{k}  \) and \( y \) to \( x_{k}  \) respectively.
\end{proof}
\noindent \textbf{Q4.} Let \( G \) be a connected, 3-regular graph. Show that if \( G \) has no cut-edge, then every pair of edges lie on a common cycle.
\begin{proof}
Let \( e = (e_1, e_2) \) and \( f = (f_1, f_2) \) be two edges in \( G \). If there exist two vertex disjoint paths with ends in \( \{ e_1, e_2 \}  \) and \( \{ f_1, f_2 \}  \), then these paths and the two edges form the needed cycle. Otherwise, Menger's theorem implies that there is a separation \( (A,B) \) of order 1 with \( \{ e_1, e_2 \} \subseteq A \) and \( \{ f_1, f_2 \} \subseteq B \). Let \( A \cap B = \{ v \}  \) and note that \( \deg v = 3 \). If \( v_1, v_2, v_3 \) are the neighbours of \( v \), assume without loss of generality that \( v_1, v_2 \in A \) and \( v_3 \in B \). Then \( (v_2, v_3) \) is a cut-edge, so this case can not occur; thus \( e \) and \( f \) are contained in a cycle.
\end{proof}
\noindent \textbf{Q5.} For any graph \( G \), \( \nu (G) \leq \tau (G) \leq 2 \nu (G) \).
\begin{proof}
We first show that \( \nu (G) \leq \tau (G) \). Given a maximal matching \( M \), let \( X \) be a set consisting of one end from each edge in \( M \). Then \( |X| = |M| \) and \( X \) is a vertex cover by the maximality of \( M \). Thus, \( \tau (G) \leq |X| = \nu (G) \).

To show that \( \tau (G) \leq 2 \nu (G) \), let \( M \) be a maximal matching in \( G \). Let \( X \) be the set of ends of edges in \( M \). Then \( X \) is a vertex cover: if there was an edge \( e \in E \) with no ends in \( X \), then \( M \) is not maximal since \( M \cup \{ e \}  \) is a matching. Hence \( \tau (G) \leq |X| = 2 |M| \leq 2 \nu (G) \).
\end{proof}
\noindent \textbf{Q6.} Prove K\"onig's theorem using Menger's theorem: \( \tau (G) = \nu (G) \) for any bipartite graph \( G \).
\begin{proof}
We may assume that \( G \) has vertices no degree-zero vertices, as they don't change anything. It suffices to prove that \( \tau (G) \leq \nu (G) \), given Q5. Suppose \( G \) is bipartite with bipartition \( (A,B) \). Let \( G' \) be obtained from \( G \) by appending two vertices \( u,v \) such that \( u \) is adjacent to every vertex in \( A \) and \( v \) is adjacent to every vertex in \( B \). Let \( k \geq 0 \) be the greatest integer for which there exist \( k \) (internally) vertex disjoint paths between \( u \) and \( v \). This \( k \) corresponds to a matching: each path contains an unique edge with an end in \( X - Y \) and the other in \( Y - X \); the set of such edges is a matching of size \( k \).

Now observe that Menger's theorem implies that there is a separation \( (X, Y) \) of \( G' \) such that \(  \in X - Y \), \( v \in Y - X \), and \( |X \cap Y| \leq k \). Then \( X \cap Y \) is a vertex cover as there can be no edge \( e \in E(G') \) with no end in \( X \cap Y \). Otherwise, one of the following must occur, a contradiction: 
\begin{itemize}[nolistsep]
	\item \( e \) has both ends in \( X \) or both ends in \( Y \), but this can not occur as \( G \) is bipartite.
	\item \( e \) has \( u \) or \( v \) as one of its ends, but this can not occur as otherwise the other end of \( e \) would need to have degree 0.
	\item \( e \) connects \( X - Y \) to \( Y - X \), but this would mean that \( (X,Y) \) is not a separation.
\end{itemize}
Thus, \( \tau (G) \leq |X \cap Y| = k \leq \nu (G) \).
\end{proof}
\noindent \textbf{Q7.} Show that a matching \( M \) is maximal if and only if \( G \) contains no \( M \)-augmenting path.
\begin{proof}
If \( P \) is an \( M \)-augmenting with \( m \) edges, note that an odd number of edges in \( P \) are in \( M \). In particular, \( \frac{m-1}{2}  \)  edges are in \( M \), and \( \frac{m+1}{2}  \) are not. Since the ends of \( P \) are unmatched, the edges in \( M \) not on \( P \) together with the edges on \( P \) not in \( M \) form a matching with more edges than \( |M| \), a contradiction.

Now suppose \( M \) is a matching in \( G \) with no \( M \)-augmenting path. Assume for a contradiction that \( M' \) is a larger matching than \( M \). We define a new graph \( G' \) by \( V(G') = V(G) \) and \( E(G') = M \cup M' \). Then since every vertex in \( G \) has degree 2, every component of \( G' \) is a path or a cycle. Since \( |M'| > |M| \), there is a component \( C \) with more edges in \( M' \) than in \( M \), i.e. \( |E(C) \cap M'| > |E(C) \cap M| \).
\begin{itemize}[nolistsep]
	\item If \( C \) is a path, then one of its ends is matched by \( M \) (else \( C \) is \( M \)-augmenting). So the first edge of \( C \) is in \( M \). From there, the edges alternate between \( M' \) and \( M \), so \( |E(C) \cap M'| \leq |E(C) \cap M| \).
	\item If \( C \) is a cycle, then it has an even number of edges (else two edges in \( M \) or \( M' \) will share a vertex as an end), so \( |E(C) \cap M'| = |E(C) \cap M| \).
\end{itemize}
Thus, there can be no such component \( C \), contradicting \( |M'| > |M| \). Thus \( M \) is maximal.
\end{proof}
\noindent \textbf{Q8.} For an integer \( k \geq 3 \), let \( N = R_{3}(k,k,k)  \) be the minimum \( N \) such that in every edge-coloring of \( K_{N}  \) in 3 colors there is a set \( X \) of \( k \) vertices so that all edges between vertices of \( X \) have the same color. Prove that \[ \binom{N}{k} \left ( \frac{1}{3}  \right ) ^{\binom{k}{2} - 1} \geq 1 \tag{$\ast$}. \]
\begin{proof}
	Suppose \[  \binom{N}{k} \left ( \frac{1}{3}  \right ) ^{\binom{k}{2} - 1} < 1 \] for a contradiction. Let \( \Omega \) be the space of all colorings of the edges of \( K_{N}  \) in 3 colors. Let \( R \subseteq V(K_{N} ) \) be a \( k \)-element subset of vertices, and let \( A_R \) be the event where \( R \) is monochromatic. So \[ \mathbb{P}(A_{R}) = 3 \prod_{1 \leq i \leq \binom{k}{2} }^{} \frac{1}{3}  =  3^{1-\binom{k}{2} }. \] By sub-additivity, \[ \mathbb{P}\left ( \bigcup_{R \in [V(K_{N})]^{k} }^{} A_{R}  \right ) \leq \sum_{R \in [V(K_{N})]^{k}}^{} 3^{1-\binom{k}{2}} = \binom{N}{k} 3^{1 - \binom{k}{2} } < 1.  \] Thus there is a non-zero probability that there is no monochromatic \( k \)-element subgraph of \( K_{N}  \), contradicting \( N = R_{3} (k,k,k) \). Hence \((\ast)\) follows and we're done.
\end{proof}
\noindent \textbf{Q9.} Let \( F \) be a forest on \( n \) vertices. Prove that the intersection of \( k \) connected subgraphs of \( F \) is either empty or a tree.
\begin{proof}
It suffices to prove the claim when \( k = 2 \), since given connected subgraphs \( C_1, C_2, \hdots , C_{k}  \), we have by induction that \(C = C_1 \cap \cdots \cap C_{k-1}  \) is either a tree or empty. If \( C \) is a tree then it is connected and we may apply the case when \( k = 2 \) to \( C \cap C_{k}  \). Otherwise \( C \) is empty so that \( C \cap C_{k}  \) is too. Thus, we just need to prove the base case now.

Let \( C_1 \) and \( C_2 \) be two connected subgraphs of \( F \). Assume that \( C_1 \cap C_2 \) is not a tree and non-empty. Since \( F \supseteq C_1 \cap C_2 \) is acyclic, so is \( C_1 \cap C_2 \). Thus, \( C_1 \cap C_2 \) cannot be connected. Since \( C_1 \cap C_2 \neq \emptyset  \), there are vertices \( u,v \) in \( C_1 \cap C_2 \) which have no path between them. Let \( C \) be the connected component containing \( \{ u,v \}  \). Since \( F \) is a forest, there is a unique path \( P \) between \( u \) and \( v \) in \( F \). Since \( u,v \in V(C_1) \cap V(C_2)\) and these are connected subgraphs, it follows that \( C_1 \cap C_2 \) contains \( P \), a contradiction.
\end{proof}
\noindent \textbf{Q10.} Let \( G \) be a \( k \)-connected graph on \( n \) vertices. 
\begin{enumerate}[(a)]
	\item Prove that \( |E(G)| \geq kn/2 \).
	\item Show that for every integer \( k \geq 2 \) and \( n \geq k + 1 \) there is a \( k \)-connected graph with \( |V(G)| = n \) and \( |E(G)| \leq (k-1)n \).
\end{enumerate}
\begin{proof}
	For (a) note that since \( G \) is \( k \)-connected, every vertex has degree at least \( k \). Otherwise, there is a vertex with degree at most \( k - 1 \); deleting its neighbours disconnects the graph, contradicting \( k \)-connectivity. By handshaking, \[|E(G)| = \frac{1}{2}  \sum_{v \in V(G)}^{} \deg v \leq nk/2.  \] 

We prove (b) by induction on \( k \). For \( k = 2 \) and \( n \geq k + 1 \), the cycle \( C_{n}  \) on \( n \) vertices is the graph we need. Indeed, \[ |E(C_{n})| = n = 1 \cdot n \leq (k-1)\cdot n. \] Now fix \( k\geq 3 \). By the IH we obtain a graph \( G' \) on \( n' \) vertices and \( m' \) edges such that \( G' \) is \( (k-1) \)-connected, \( n' \geq k \), and \( m' \leq (k - 2)n' \). Now let \( G \) be obtained by taking a vertex \( v \in V(G') \) and connecting it to every vertex in \( G' \). Then from Q1 \( G \) is \( k \)-connected. This uses \( n' \) edges. Let \( n = n' + 1 \) and \( m = m' + n' \). Then \( n = n' + 1 \geq k + 1 \) and
	\begin{align*}
		m &= m' + n' \leq (k-2)n' + n' = (k-1)n' \\ 
		  &\leq (k-1)(n' + 1) = (k-1)n
	\end{align*}
which completes the proof.
\end{proof}
\noindent \textbf{Q11.} Prove that:
\begin{enumerate}[(a)]
	\item If \( T \) is a tree then \( |V(T)| = |E(T)| + 1 \).
	\item If \( F \) is a forest and \( c(F) \) is the number of components of \( F \), then \( c(F) = |V(F)| - |E(F)| \).
\end{enumerate}
\begin{proof}
For (a) we proceed by induction on \( |V(T)| \). If \( |V(T)| = 1 \) then \( T \) is edgeless so that \( |V(T)| = 0 + 1 = |E(T)| + 1 \). Now fix \( |V(T)| \geq 2 \).  Then \( T \) has at least one leaf \( v \). Let \( T' = T \setminus v \). Then \( T' \) is a tree on \( |V(T)| - 1 \) vertices, so by the IH we have \( |V(T')| = |E(T')| + 1 \). Appending \( v \) to \( T' \) gives one new vertex and one new edge. Hence \( |V(T)| = |V(T')| + 1 = |E(T')| + 1 + 1 = |E(T)| + 1 \) and we're done.

For (b) we induct on \( c(F) \). If \( c(F) = 1 \) then from (a) \[|V(F)| = |E(F)| + 1 \Leftrightarrow   1 = |V(F)| - |E(F)|. \] Then if \( c(F) \geq 2 \), let \( F' \) be obtained from \( F \) by deleting one entire component \( C \). Then by the IH we have \( c(F') = |V(F')| - |E(F')| \) and since \( C \) is a tree, (a) implies that \( |V(C)| = |E(C)| + 1  \). Putting these two together, we obtain that
\begin{align*}
	c(F) &= c(F') + 1 = |V(F')| - |E(F')| + |V(C)| - |E(C)| \\
	     &= |V(F)| - |E(F)|, 
\end{align*}
since \( V(F') \cap V(C) = E(F') \cap E(C) = \emptyset  \). This completes the proof. 
\end{proof}
\noindent \textbf{Q12.} Prove Hall's theorem: Let \( G \) be bipartite with bipartition \( (A,B) \). Show that \( G \) has a matching which saturates \( A \) if and only if \[ |N(S)| \geq |S| \] for every subset \( S \subseteq A \).
\begin{proof}
The forward implication is easy: suppose \( M \) is a matching which saturates \( A \). If \( S \subseteq A \)  were such that \( |N(S)| < |S| \), then no matching could saturate \( S \) and hence \( A \).

Conversely, first note that Hall's condition implies that \( |A| \leq |B|\). Without loss of generality, we assume \( |A| = |B| \). Thus, a matching \( M \) is perfect if and only if it saturates \( A \). So let \( M \) be a matching in \( G \); we will show that either \( G \) has an \( M \)-augmenting path or there is a set \( S \subseteq A \) violating Hall's condition. Then, the contrapositive implies Hall's theorem if we take \( M \) to be maximum.

If \( M \) does not already saturate \( A \), there is an unmatched vertex \( a_0 \in A \) with a neighbour \( b_1 \in B \). If \( b_1 \) is unmatched, then the path with the single edge \( a_0b_1 \) is augmenting. Otherwise, there is a vertex \( a_1 \in A \) which is matched with \( b_1 \) by \( M \). Note that we have constructed an alternating path with ends \( a_0 \) and \( a_1 \). 

We continue this process inductively as follows. Let \( k \geq 2 \) and suppose we have defined \( \{ a_0, a_1, \hdots , a_{k - 1}  \}  \) and \( \{ b_1, b_2, \hdots , b_{k - 1}  \}  \) such that there is an alternating path with ends \( a_0 \) and \( a_{j}  \), for each \( j \in \{ 0, 1, \hdots ,k-1 \}  \). Then exactly one of the following holds:
\begin{itemize}[nolistsep]
	\item If \( S = \{ a_0, a_1, \hdots , a_{k-1}  \}  \) satisfies Hall's condition, then there is a vertex \( b_{k} \notin \{ b_1, b_2, \hdots ,b_{k-1}  \}   \) adjacent to a vertex in \( S \). If \( b_{k}  \) is unmatched, then there is an \( M \)-augmenting path. Otherwise, \( b_{k}  \) is matched with a vertex \( a_{k} \in A \) and so there is an alternating path between \( a_0 \) and \( a_{k}  \).
	\item Else, \( S \) violates Hall's condition and the process terminates.
\end{itemize}
Since this process must eventually terminate, we either halt with an augmenting path or a subset \( S \) violating Hall's condition.
\end{proof}
\noindent \textbf{Q13.} Prove K\"onig's theorem using Hall's theorem.
\begin{proof}
Again, as in Q6, we first notice that it suffices to show that \( \tau (G) \leq \nu (G) \). Let \( G \) have bipartition \( (A,B) \) and let \( M \) be a maximal matching in \( G \). If \( M \) saturates \( A \), then \( \nu (G) = |M| = |A| \) and \( A \) is a vertex cover so that \( \tau (G) = \nu (G) \). Thus we assume that \( L \subsetneq A \) is the largest subset of \( A \) saturated by \( M \). Hence all vertices in \( A - L\) are unmatched. 

By induction on \( |A - L| \), we show that a vertex cover of size \( |L| \) exists. We note that if \( |A - L| \geq 1 \) then there is a subset \( S \subseteq A - L \) with \( |N(S)| < |S| \). Indeed, it follows from Hall's theorem that since \( A \) can not be saturated by a matching, there is a subset \( S \subseteq A \) violating Hall's condition. Let's take \( |S| \) minimum over all such sets \( S \). Note then that \( S \cap L = \emptyset  \) since every vertex in \( L \) has at least one unique neighbour.

For the base case step, if \( |A - L| = 1 \) there is a unique unmatched vertex \( a \in A - L \). So \( a \) has no neighbours since \( |N(\{ a \})| < 1 \). Thus \( L \) is a vertex cover. Furthermore, there is a maximal matching \( M \) in \( G \) such that every edge of \( M \) has exactly one end in \( L \), and where every vertex in \( A - L \) is still unmatched.

Now fix \( |A - L| \geq 2 \) and obtain a set \( S \subseteq A - L \) which violates Hall's condition. Then the IH implies that \( H = G \setminus (S \cup N(S)) \) contains a vertex cover \( X \) of size \( |L| \). So, in \( G \), every edge with no end in \( N(S) \) has an end in \( X \). Also, there is a maximal matching \( M' \) in \( G \) such that each edge in \( M' \) has exactly one end in \( X \) and no vertex in \( A - L \) is matched by \( M' \).

Note that since every vertex in \( S \subseteq A - L\) is unmatched by \( M' \), every vertex in \( N(S) \) is matched, otherwise there is an augmenting path. Define \( Y \) to be the set of vertices in \( X \) which are not matched with a vertex in \( N(S) \). Then we claim that \( Y \cup N(S) \) is a vertex cover of \( G \) of size at most \( |L| \).
\begin{itemize}
	\item First, we prove that \( Y \cup N(S) \) is a vertex cover. Let \( e \) be any edge in \( G \). If \( e \in M' \) then it has one end \(x \in  X \). If \( x \notin Y \), then by definition \( x \in N(S) \), as needed.
	\item Next, we show that \( |Y \cup N(S)| \) has size at most \( |L| \). By construction, every vertex in \( X \setminus Y \) has a unique end in \( N(S) \), so \( |X \setminus Y| \leq |N(S)| \). Also, since every vertex \( x \in N(S) \) is matched by \( M' \), either \( x \in X \setminus Y \) (if \( x \in X \)) or \( x \) has a unique neighbour \( y \in X \setminus Y \), so \( |N(S)| \leq |X \setminus Y| \). Since \( Y \cap N(S) = \emptyset  \), we have \( |Y \cup N(S)| = |Y| + |N(S)| = |Y| + |X \setminus Y| = |X| = |L|. \) 
\end{itemize}
Therefore, \( G \) has a vertex cover of size \( |L| \). Therefore, \( \tau (G) \leq |L| = \nu (G) \).
\end{proof}
\end{document}
