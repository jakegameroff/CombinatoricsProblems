%! TeX root: ../main.tex
\noindent \textbf{Q14.} Prove Tutte's matching theorem: A graph \( G \) has a perfect matching if and only if \[ c_{o} (G - X) \leq |X| \] for every subset \( X \subseteq V(G) \).
\begin{proof}
Let's start with the ``only if" direction. Let \( M \) be a perfect matching in \( G \), and suppose there is a set \( X \subseteq V(G) \) with \( c_{o}(G - X) > |X|  \). Suppose \( C_1, C_2, \hdots , C_{k}  \) are the odd components of \( G - X \). Since \( M \) is perfect and each component is odd, for each \( j \in [k] \) there is a vertex \( v_{j} \in V(C_{j}) \) which is matched with a vertex \(x_{j} \notin C_{j}  \); hence \( x_{j} \in X \). But then two vertices \( v_{i_1} , v_{i_2}  \) must receive the same match in \( X \), a contradiction.

The ``if" direction is much harder. We proceed by induction on \( |V(G)| \). The theorem is trivial if \( |V(G)| \leq 2 \), so fix \( |V(G)| \geq 3 \) and suppose \( c_{o} (G - X) \leq |X| \) for every \( X \subseteq V(G) \). We prove a sequence of claims as follows:
\begin{itemize}[nolistsep]
	\item \textbf{Claim 1:}	\emph{\( |V(G)| \) is even.} It suffices to check that \( G \) has only even components. Set \( X = \emptyset  \). Then \( c_{o} (G - X) = c_{o} (G) \leq |\emptyset | = 0 \), so the claim holds.
	\item \textbf{Claim 2:} \emph{\( c_{o} (G - X) + |X| \) is always even.} If \( |X| \) is odd, then \( |V(G)| - |X| \) is too by claim 1, so \( G - X \) must have an odd number of components. The exact same reasoning shows that if \( |X| \) is even then so is \( c_{o} (G - X) \). Then, the claim holds since \( a + b \) is even if and only if \( a \) and \( b \) have the same parity.
	\item \textbf{Claim 3:} \emph{There is a subset \( X \subseteq V(G) \) such that \( c_{o} (G - X) = |X| \).} We will call such sets \( X \) \emph{critical}. If \( X = \emptyset  \) then \( c_{o} (G - X) = 0  \) as shown above, so \( \emptyset  \) is critical.
	\item \textbf{Claim 4:} \emph{Let \( Z \subseteq V(G) \) be critical with \( |Z| \) maximum. Then \( G - Z \) has no even components.} Suppose by contradiction that \( C \) is an even component of \( G - Z \), and fix \( x \in V(C) \). Then \( Z' = Z \cup \{ x \}  \) is critical so that \( Z \) is not maximal: \( c_{o} (G - Z') = c_{o} (G - Z) + 1 = |Z| + 1 = |Z'|  \), since deleting \( x \) from \( C \) will either give one odd component (if \( C - x \) is connected) or one even and one odd component.
	\item \textbf{Claim 5:} \emph{For each \( j \in [k] \), fix \( v_{j} \in V(C_{j}) \). Then \(C_{j}^{\ast} = C_{j} - v_{j}  \) has a perfect matching.} Suppose not. Then the IH implies that there is a set \( X \subseteq V(C^{\ast} _{j}) \) with \( c_{o} (C^{\ast} _{j} - X) > |X| \). But then \( Z' = Z \cup X \cup \{ v_{j}  \}  \) is critical:
		\begin{align*}
			c_{o} (G - Z') &= c_{o} (G - Z) - 1 + c_{o} (C^{\ast} _{j}  - X  ) \\
				       &> c_{o}(G - Z)  - 1 + |X| \\
				       &= |Z| - 1 + |X| = |Z'| - 2,
		\end{align*}
		so \( c_{o} (G - Z') \geq |Z'| - 1 \), but \( c_{o} (G - Z') \) and \( |Z'| \) have the same parity by claim 2, so we must have \( c_{o} (G - Z') \geq |Z'| \). By hypothesis, \( c_{o} (G - Z') \leq |Z'| \), so \( Z' \) is critical.
	\item \textbf{Claim 6:} \emph{\( G \) has a perfect matching.} Claim 5 shows that \( C^{\ast} _1, C^{\ast} _2, \hdots , C^{\ast} _{k}  \) have perfect matchings; so now we must match the points in \( Z \) with points in \(Y = \{ v_1, v_2, \hdots , v_{k}  \}  \). If there is no such (perfect) matching between \( Z \) and \( Y \), then Hall's theorem implies that there is a subset \( S \subseteq Y \) with \( |N(S)| < |S| \) (note that the induced graph is bipartite, deleting any edges \( z_{i} z_{j}  \)). Set \( X = N(S) \) and then note that \( c_{o} (G - X) \geq |S| > |N(S)| = |X| \) is a contradiction and the proof is done. \qedhere
\end{itemize}
\end{proof}
\noindent \textbf{Q15.} Let \( G \) be a \( d \)-regular bipartite graph. Show that \( G \) has a perfect matching.
\begin{proof}
Suppose not. Let \( (A,B) \) be a bipartition of \( G \). Then Hall's theorem says that there is a set \( S \subseteq A \) with \( |N(S)| < |S| \). Note that there are \( d \cdot |S| \) edges leaving \( S \), and there are \( d \cdot |N(S) \) edges leaving \( N(S) \). But every edge leaving \( S \) has another end in \( N(S) \), so \( d \cdot |N(S)| < d \cdot |S| \) is a contradiction.
\end{proof}
\noindent \textbf{Q16.} Given \( n \in \mathbb{N}  \), determine the minimum \( \delta = \delta (n) \) such that every graph \( G \) on \( 2n \) vertices with minimum degree \( \delta  \) has a perfect matching.
\begin{proof}
First, we show that \( \delta = n \) suffices to guarantee a perfect matching in \( G \). We may assume that \( G \) is \( \delta  \)-regular, otherwise just delete any extra edges.

Given any graph \( G \), let \( (A_0,B_0) \) be any partition of \( V(G) \) such that \( |A_0| = |B_0| \). For any partition \( (A,B) \), let \( e(A) \) denote the number of edges with both ends in \( A \), and define \( e(B) \) analagously. We give an algorithm to obtain a bipartite subgraph of \( G \) which will contain a perfect matching.

Pick any \( a \in A_0 \) and \( b \in B_0 \). Define \( A_1 = A_0 \setminus \{ a \} \cup \{ b \}  \) and \( B_1 = B_0 \setminus \{ b \} \cup \{ a \}  \). If \( e(A_0) + e(B_0) > e(A_1) + e(B_1) \), then repeat the algorithm again with the partition \( (A_1, B_1) \), otherwise try again with any other pair of vertices. Since \( G \) is finite, this process terminates with a partition \( (A,B) \) of \( V(G) \) which minimises the number of edges with both ends in \( A \) or both ends in \( B \). Then, it follows that each \( a \in A \) has at least \( \lceil n / 2 \rceil \) neighbours in \( B \) and each \( b \in B \) has at least \( \lceil n / 2 \rceil \) neighbours in \( A \). Let \( H \) be obtained from \( G \) by deleting all edges with both ends in either \( A \) or \( B \). Then \( H \) is \( \lceil n/2 \rceil \)-regular and bipartite, and so by Q15 it has a perfect matching. Thus, \( G \) does too.

We complete the proof by showing that \( \delta = n - 1 \) does not suffice to guarantee a perfect matching. Consider the complete bipartite graph \( K_{n-1, n+1}  \). Deleting the smaller partite set of \( n - 1 \) vertices results in \( n + 1 \) isolated vertices, which violates Tutte's condition. Hence, \( K_{n-1, n+1}  \) has no perfect matching. Note that \( \delta (K_{n-1, n+2}) = n - 1 \).
\end{proof}
\noindent \textbf{Q17.} Show that for every bridgeless cubic graph \( G \) and every \( e \in E(G) \) there is a perfect matching in \( G \) containing \( e \).
\begin{proof}
Fix any edge \( uv \in E(G) \) and let \( e,f \in E(G) \) be the other two edges incident to \( v \). Let \( G' \) be obtained from \( G \) by deleting the edges \( e,f \). To show that \( G \) has a perfect matching containing \( uv \), it suffices to prove that \( G' \) has a perfect matching. Hence, we use Tutte's theorem to show that \( c_{o} (G' - X) \leq |X| \) for every subset \( X \subseteq V(G') \).

Suppose for a contradiction that there is a set \( X \subseteq V(G) \) such that \( c_{o} (G' - X) > |X| \). Note that \( c_{o} (G' - X) \) and \( |X| \) have the same parity since \( |V(G')| \) is even, hence \( c_{o} (G' - X) \geq |X| + 2 \). Let \( C_1, C_2, \hdots , C_{n}  \) denote the odd components of \( G' - X \). For each \( j \in [n] \), let \( \ell_{j}  \) be the number of edges in \( G \) leaving \( C_{j}  \). Then since \( G \) is cubic, \[ \sum_{v \in V(C_{j})}^{} \deg_{G} v = 2 |E(C_{j})| + \ell_{j} = 3 |V(C_{j})|. \] Observe that LHS is even (since \( |V(C_{j})| \) is odd), hence \( \ell_{j} \) must be odd. Furthermore, \( \ell_{j} \geq 2 \) since \( G \) is bridgeless, and since \( \ell_{j}  \) is odd we have \( \ell_{j} \geq 3 \). Let \( r_{j} \) be the number of edges in \( G \) with one end in \( C_{j}  \) and the other in another odd component \( C_{i}  \), and let \( q_{j} = \ell_{j} - r_{j}  \) be the remaining edges. Then \( \sum_{j=1}^{n} r_{j} \leq 4 \) since we only deleted two edges, and since \( G \) is 3-regular, \( \sum_{j=1}^{n} q_{j} \leq 3|X|  \). 

Then, putting everything together,
\begin{align*}
	c_{o} (G' - X) &= n \leq \frac{1}{3} \sum_{j=1}^{n} \ell_{j} = \frac{1}{3} \sum_{j=1}^{n} (q_{j} + r_{j}) \\
	 &\leq \frac{1}{3}\sum_{j=1}^{n} q_{j} + 4/3 \leq |X| + 4/3
\end{align*}
But recall that \( c_{o} (G' - X) = n \geq |X| + 2\), so we may rearrange the above to obtain \( |X| + 2 - 2/3 \geq n \) and hence \( n - 2/3 \geq n \) is a contradiction. Therefore, \( G' \) has a perfect matching. Since \( v \) has degree one in \( G' \), it follows that \( G \) has a perfect matching containing \( uv \). 
\end{proof}
\noindent \textbf{Q18.} Prove the Tutte-Burge formula: For a graph \( G \), its \emph{deficiency} \( \operatorname{def} (G) \) is the minimal number of vertices avoided by a matching. Clearly \( \operatorname{def} (G) = |V(G)| - 2 \nu (G) \). Show that \[ \operatorname{def} (G) = \max_{X \subseteq V(G)} (c_{o} (G - X) - |X|). \tag{$\ast$} \] 
\begin{proof}
First note that \((\ast)\) holds if \( G \) has a perfect matching: \( c_{o} (G - X) - |X| \leq 0 \) always by Tutte's theorem, and the inequality is tight taking \( X = \emptyset  \). Therefore, \[ \operatorname{def} (G) = |V(G)| - 2 \nu (G) = 0 =  \max_{X \subseteq V(G)} (c_{o} (G - X) - |X|).  \] 

Let \( X \subseteq V(G) \) be such that \( c_{o} (G - X) - |X| = k \) is maximum. Let \( M \) be any maximal matching in \( G \) and let \( C_1, C_2, \hdots , C_{n}  \) be the odd components of \( G - X \). By deleting edges, we may assume without loss of generality that \( G \) has no edge with one end in \( \medcup_{j=1} ^{k} V(C_{j}) \) and another in \( X \), since at most \( n - k \) edges in \( M \) can have an end in \( X \). Then each component \( C_1, C_2, \hdots , C_{k}  \) contains an unmatched vertex, since \( |V(C_{j})| \) is odd for every \( j \in [k] \). So \( k \) vertices can be avoided by a matching, hence \[\operatorname{def} (G) = |V(G)| - 2\nu (G) \leq k = \max _{X \subseteq V(G)} (c_{o} (G - X) - |X|). \] 

The reverse inequality is much harder. Let's set \( k = |V(G)| - 2\nu (G) \). We construct an auxiliary graph \( H \) as follows. Let \( Y = \{ y_1, y_2, \hdots , y_{k}  \}  \) be a set of \( k \) new vertices. Let \( H = G \cup Y \) be obtained from \( G \) by adding each vertex in \( Y \), with each \( y_{j} \in Y \) adjacent to every other vertex in \( H \) (including those in \( Y - v_{j}  \)). 

Then \( H \) has a perfect matching: take any maximal matching \( M \) in \( G \) and suppose \( v_1, v_2, \hdots , v_{k}  \) are unmatched; then \( M \cup \{ y_{j}v_{j} : j \in [k] \}  \) is a perfect matching in \( H \). So \( H \) satisfies Tutte's condition: we have \( c_{o} (H - X) \leq |X| \) for every \( X \subseteq V(H) \). So fix \( X \subseteq V(G) \) and observe that
\begin{align*}
	c_{o} (G - X) &= c_{o} (H - (X \cup Y)) \leq |X \cup Y| \\ 
		      &= |X| + |Y| = |X| + |V(G)| - 2 \nu (G),
\end{align*}
consequently \( c_{o} (G-X) - |X| \leq |V(G)| - 2 \nu (G) = \operatorname{def} (G) \). Since \( X \) was arbitrary, the proof is complete.
\end{proof}
\noindent \textbf{Q19.} Use LP duality to prove K\"onig's theorem.
\begin{proof}
Fix a bipartite graph \( G = (V, E) \). First note that the following LP computes \( \nu (G) \):
\begin{align*}
    \text{max} \ \ \ & z(x) = \sum_{e \in E}^{} x_{e}  \\
    \text{subject to} \ \ \ & \sum_{e \in \delta (v) }^{} x_{e} \leq 1, \ \forall v \in V \\
			    & x_{e} \geq 0, \ \forall e \in E.
\end{align*}
Since \( G \) is bipartite, its incidence matrix \( A \) is totally unimodular. Hence the polyhedron \( P = \{ x \geq 0 : Ax \leq 1 \}  \) has integral corners. By construction, \( x \in P \) if and only if it is feasible, so any optimum \( x^{\ast}  \) is integral.

Now fix a maximal matching \( M \) and define \( x_{e} = 1 \) if \( e \in M \) and \( x_{e} = 0 \) otherwise. Then \( x \) is a feasible point and \( z(x) = |M| = \nu (G) \). On the other hand, if \( x^{\ast}  \) is an optimum of the LP, put \( M = \{ e \in E : x^{\ast} _{e} = 1 \}  \). Then \( |M| = \sum_{e \in E}^{} x^{\ast} _{e}  \), and if \( M \) is not a matching there is a vertex \( v \in V \) incident with two edges in \( M \), violating the first constraint. So the primal has optimum value \( \nu (G) \).

Consequently, LP duality implies that its dual program also has optimum value \( \nu (G) \). The dual is as follows:
\begin{align*}
    \text{min} \ \ \ & w(x) = \sum_{v \in V}^{} y_{v}   \\
    \text{subject to} \ \ \ & y_{u} + y_{v} \geq 1, \ \forall uv \in E \\
			    & y_{v} \geq 0, \ \forall v \in V.
\end{align*}
But the dual also computes \( \tau (G) \). If \( X \) is a minimal vertex cover of \( G \), set \( y_{v} = 1 \) if \( v \in X \) and \( y_{v} = 0 \) otherwise. Then \( y \) is a feasible point and \( w(y) = |X| \). Conversely, if \( y^{\ast}  \) is an optimum of the LP, put \( X = \{ v \in V : y^{\ast}_{v} = 1 \}  \). If \( X \) is not a vertex cover, then there is an edge \( uv \in E \) with no end in \( X \). So \( y^{\ast} _{u} = y^{\ast} _{v} = 0  \), contradicting \( y^{\ast}_{u} + y^{\ast} _{v} \geq 1  \). It follows that \( \nu (G) = \tau (G) \).
\end{proof}
