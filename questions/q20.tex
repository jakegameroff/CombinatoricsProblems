%! TeX root: ../main.tex
\noindent \textbf{Q20.} Let \( G \) be a graph with minimum degree \( \delta \geq 2 \). Show that \( G \) contains a cycle of length at least \( \delta + 1 \).
\begin{proof}
Let \( P \) be a path in \( G \) with maximum length, and let \( u \) be an end of \( P \). By the maximality of \( P \), each neighbour of \( u \) is in \( P \), so \( P \) has length at least \( \delta \). Let \( v \) be the furthest vertex along \( P \) from \( u \) such that \( uv \in E \), and let \( P' \subseteq P \) be a sub-path of \( P \) with ends \( u,v \). Then \( P' \) has length at least \( \delta \), so \( P' \cup uv \) is a cycle of length at least \( \delta + 1 \).
\end{proof}
\noindent \textbf{Q21.} Let \( G \) be a bipartite graph with maximum degree \( \Delta \). Prove that \( G \) has a matching saturating every vertex of degree \( \Delta \).
\begin{proof}
Let \( (A,B) \) be a bipartition of \( G \). By appending isolated vertices, we may assume \( |A| = |B| \). Let \( H \) be a bipartite \( \Delta \)-regular graph obtained from \( G \) by adding a set \( F \) of extra edges. Then Q15 implies that \( H \) has a perfect matching \( M \) and so \( M - F \) is the desired matching in \( G \).
\end{proof}
\noindent \textbf{Q22.} Let \( \alpha (G) \) and \( \rho (G) \) denote the independence and edge-cover numbers of \( G \) respectively. Show that if \( G = ( A \cup B, E) \) is bipartite, then \( \alpha(G) = \rho (G) \).
\begin{proof}
Let's use LP duality as in Q19. We'll start by writing an LP which computes \( \rho (G) \). Here it is:
\begin{align*}
    \text{min} \ \ \ & \sum_{e \in E}^{} x_{e}  \\
    \text{subject to} \ \ \ & \sum_{e \in \delta (v) }^{} x_{e} \geq 1, \ \forall v \in V \\
			    & x_{e} \geq 0, \ \forall e \in E.
\end{align*}
Using the same reasoning as in Q19, since \( G \) is bipartite, its incidence matrix \( A \) is totally unimodular. Then the set \( P = \{ x \geq 0 : Ax \geq 1 \}  \) of feasible solutions has integral corners and so the LP is integral.

First suppose \( F \) is a minimal edge cover in \( G \). Then set \( x_{e} = 1 \) if \( e \in F \) and \( x_{e} = 0 \) otherwise. Then \( x \) is feasible by construction. On the other hand, given an optimum \( x^{\ast}  \), note first that \( x^{\ast}_{e} \in \{ 0,1 \}  \) for each \( e \in E \). Indeed, the LP is integral, and if there is an edge \( e = uv \in G \) with \( x_{e}^{\ast} \geq 2  \), it follows that \( x^{\ast}  \) is not optimal. So put \( F = \{ e \in E : x^{\ast} _{e} = 1 \}  \). Then \( \rho (G) \leq |F| = \sum_{e \in E}^{} x_{e}^{\ast}   \), and note that \( F \) is a vertex cover, otherwise there is a vertex \( v \in V \) incident to no edge in \( F \). But then \( \sum_{e \in \delta (v)}^{} x^{\ast}_{e} = 0  \) violates the LP constraints. Hence the primal computes \( \rho (G) \).

Using LP duality, the dual LP must then also compute \( \rho (G) \). Here it is:
\begin{align*}
    \text{max} \ \ \ & \sum_{v \in V}^{} y_{v}   \\
    \text{subject to} \ \ \ & y_{u} + y_{v} \leq 1, \ \forall uv \in E \\
			    & y_{v} \geq 0, \ \forall v \in V.
\end{align*}
It remains to show that the dual LP computes \( \alpha(G) \). First, given a maximum independent set \( X \subseteq V \), put \( y_{v} = 1 \) if \( v \in X \) and \( y_{v} = 0 \) otherwise. Then \( y \) is feasible. On the other hand, given an optimum \( y^{\ast}  \), put \( X = \{ v \in V : y_{v} = 1 \}  \). Then \( \alpha (G) \geq |X| = \sum_{v \in V}^{} y_{v}  \) and \( X \) is a vertex cover, otherwise there is an edge between two vertices \( u, v \in X \). Consequently, \( y_{u} + y_{v} = 2 \) is a contradiction. Therefore, \( \rho (G) = \alpha (G) \).
\end{proof}
\noindent \textbf{Q23.} Prove a generalized version of Menger's theorem: Given \( k \in \mathbb{N}  \) and \( S, T \subseteq V \), there are either \( k \) (vertex) disjoint paths between \( S \) and \( T \) or there is a set \( C \subseteq V \) such that every path between \( S \) and \( T \) intersects \( C \) and \( |C| < k \).
\begin{proof}
Let \( D = (V, A) \) be any digraph and first suppose that there are not \( k \) disjoint paths between \( S \) and \( T \). Let \( m < k \) be the maximum number of disjoint paths between \( S \) and \( T \). Given a collection of \( m \) such paths, let \( C \) be comprised of one vertex from each path. Then every path between \( S \) and \( T \) intersects \( C \), otherwise \( m \) is not maximal.

Conversely, suppose there is no set \( C \subseteq V\) as described above. We show that there are \( k \) disjoint paths between \( S \) and \( T \). The proof is by induction on the number of arcs \( |A| \). If \( D \) is arcless the claim is vacuously true, so fix \( |A| \geq 1 \) and an arc \( a \in A \). Let \( D' \) be obtained from \( D \) by deleting \( a \). If there are \( k \) disjoint paths in \( D' \) between \( S \) and \( T \), then the same holds in \( D \) and we're done. Otherwise, the IH implies that there is a set \( C' \subseteq V \) such that every path in \( D' \) between \( S \) and \( T \) meets \( C' \) and \( |C'| \leq k - 1 \). Then every path in \( D \) between \( S \) and \( T \) uses a vertex in \( C = C' \cup \{ u \}  \), where \( a = (u, v) \). But then by hypothesis, \( |C' \cup \{ u \}| \geq k \), so \( |C'| \geq k - 1 \), and hence \( |C| = k \). We are ready to find \( k \) disjoint paths between \( S \) and \( T \).

We first claim that in \( D' \) there are \( k \) disjoint paths \( P_1, P_2, \hdots , P_{k}  \) between \( S \) and \( C \). If not, the IH gives a set \( X \subseteq V \) such that every path between \( S \) and \( C \) intersects \( B \) and \( |B| \leq k - 1 \). But recall that every path between \( S \) and \( T \) in \( D \) intersects \( C \), so each of these paths also intersects \( B \), a contradiction. Identical reasoning implies that there are also \( k \) disjoint paths \( Q_1, Q_2, \hdots , Q_{k}  \) between \( C' \cup \{ v \}  \) and \( T \) in \( D' \).

Since \( |C| = k \), the paths \( P_1, \hdots , P_{k}, Q_1, \hdots , Q_{k} \) along with the arc \( a \) form \( k \) disjoint paths between \( S \) and \( T \). Note that no two paths $P_i$ and $Q_j$ can intersect as otherwise there is a path between $S$ and $T$ avoiding $C$. This completes the proof.
\end{proof}
